\documentclass[dvipdfmx,titlepage,10pt]{jarticle}

\usepackage[dvipdfmx]{graphicx}
\usepackage{latexsym}
\usepackage{url}
\usepackage{setspace}
\usepackage{svg}
\usepackage{here}
\usepackage{lipsum}
\usepackage{enumerate}
\usepackage{multirow}
\usepackage{booktabs}
\usepackage{threeparttable}
\usepackage{seqsplit}
% \usepackage{master-cover}
\usepackage{master_thesis}
\usepackage{master}
\usepackage[nobreak]{cite}

\usepackage{listings, jlisting} 		% for source code

\lstset{
	%プログラム言語(複数の言語に対応,C,C++も可)
 	% language = C++,
 	%枠外に行った時の自動改行
 	breaklines = true,
 	%自動改行後のインデント量(デフォルトでは20[pt])
 	breakindent = 12pt,
 	%標準の書体
 	basicstyle = \ttfamily\scriptsize,
 	%コメントの書体
 	commentstyle = {\itshape \color[cmyk]{1,0.4,1,0}},
 	%関数名等の色の設定
 	% classoffset = 0,
 	%キーワード(int, ifなど)の書体
 	% keywordstyle = {\bfseries \color[cmyk]{0,1,0,0}},
 	%表示する文字の書体
 	stringstyle = {\ttfamily \color[rgb]{0,0,1}},
 	%枠 "t"は上に線を記載, "T"は上に二重線を記載
	%他オプション:leftline,topline,bottomline,lines,single,shadowbox
 	frame = ltbr,
	%行番号の位置
	% numbers = left,
	%行番号の間隔
 	stepnumber = 1,
	%タブの大きさ
 	tabsize = 4,
}

\usepackage[a4paper,textheight=21.7cm,top=3cm,headheight=15pt]{geometry}
% \usepackage[a4paper,textheight=21.7cm,top=3cm,left=2cm,right=2cm,headheight=15pt]{geometry}


\setcounter{secnumdepth}{6}
\makeatletter
\renewcommand{\paragraph}{\@startsection{paragraph}{4}{\z@}%
   {1.5\Cvs \@plus.5\Cdp \@minus.2\Cdp}%
   {.5\Cvs \@plus.3\Cdp}%
   {\reset@font\normalsize\bfseries}}
\makeatother

\makeatletter
\newcommand{\subsubsubsection}{\@startsection{paragraph}{4}{\z@}%
  {1.0\Cvs \@plus.5\Cdp \@minus.2\Cdp}%
  {.1\Cvs \@plus.3\Cdp}%
  {\reset@font\sffamily\normalsize}
}
\makeatother
\setcounter{secnumdepth}{4}

\newcommand\figref[1]{図~\ref{#1}}
\newcommand\tabref[1]{表~\ref{#1}}



\begin{document}
\setlength{\baselineskip}{7mm} % 行間
\setlength{\oddsidemargin}{6mm}
\pagestyle{empty}

% 情報学科卒論表紙作成マクロ  Ver. 1.1                 1998.2.13
%
% このファイルには、情報学科の卒論の表紙を作成するための LaTeX の
% マクロ \makecoverpage が定義されています。以下のサンプルを参考に
% して利用して下さい。
%
%  \documentstyle[a4j]{jarticle}
%
%  \input cover.mac
%
%  % 卒研の題目が長い場合には、\\ を挿入し、行を分割することができます。
%  \卒研題目={立命館大学における \\ 情報工学に \\ 関する研究}
%  %\卒研題目={立命館大学における \\ 情報工学に関する研究}
%  %\卒研題目={情報工学に関する研究}
%  %\卒研題目={大学における情報工学に関する研究}
%
%  \氏名={草津 花子}
%  \学籍番号={2210940000-0}
%  \指導教員={立命 太郎}
%  \提出日={1998年2月20日}
%
%  % 共同研究者がいる時には次の行のコメントをはずす。
%  %\共同研究者あり
%  \共同研究者={情報 二郎 ・ 情報 三郎 ・ \\ 情報 史郎 ・ \\ 情報 五郎}
%
%  \begin{document}
%
%  \makecoverpage
%
%  \end{document}
%

\newtoks\学期
\newtoks\年度
\newtoks\クラス
\newtoks\卒研題目
\newtoks\氏名
\newtoks\回生
\newtoks\学籍番号
\newtoks\指導教員
\newtoks\提出日
\newtoks\共同研究者
\newtoks\共同研究者なし

\newif\if共同研究者あり \共同研究者ありfalse
\def\共同研究者あり{\共同研究者ありtrue}

\newdimen\X卒業論文
\newdimen\Y卒業論文
\newdimen\X題目
\newdimen\Y題目
\newdimen\X氏名
\newdimen\Y氏名
\newdimen\X共同研究者
\newdimen\Y共同研究者
\newdimen\X情報学科
\newdimen\Y情報学科
\newdimen\X情シス
\newdimen\Y情シス
\newdimen\Xoffset
\newdimen\Yoffset

\newdimen\X年度
\newdimen\Y年度


\newbox\dummybox
\newbox\titlebox
\makeatletter
\newcommand{\fontHuge}[0]{\@setfontsize\fontHuge\@xxvpt{33}}
\newcommand{\fontLARGE}[0]{\@setfontsize\fontHuge\@xviipt{25}}
\makeatother

\thispagestyle{empty}

\newcommand{\makecoverpage}[0]{
  \X卒業論文=65mm   \Y卒業論文=30mm
  \X題目=17mm       \Y題目=84mm
  \X氏名=51mm       \Y氏名=151mm
  \X共同研究者=51mm \Y共同研究者=193mm
%  \X情報学科=46.5mm \Y情報学科=224.5mm
  \X情シス=28mm     \Y情シス=224mm
  \X年度=5mm	\Y年度=0mm

% a4j のスタイルでは、\oddsidemargin=0in, \topmaigin=-0.3in で、
% \Xoffset=\Yoffset=0mm で良い。
% a4j スタイルとは異なるマージンにも対応するために、
  \Xoffset=0in    \advance\Xoffset by -\oddsidemargin
  \Yoffset=-0.3in \advance\Yoffset by -\topmargin

  \advance\X卒業論文 by \Xoffset
  \advance\Y卒業論文 by \Yoffset
  \advance\X題目 by \Xoffset
  \advance\Y題目 by \Yoffset
  \advance\X氏名 by \Xoffset
  \advance\Y氏名 by \Yoffset
  \advance\X共同研究者 by \Xoffset
  \advance\Y共同研究者 by \Yoffset
  \advance\X情報学科 by \Xoffset
  \advance\Y情報学科 by \Yoffset
  \advance\X情シス by \Xoffset
  \advance\Y情シス by \Yoffset
  
  \XY{\X年度}{\Y年度}{
  \begin{minipage}{100mm}
  {\large
  \the\年度 年度(\the\学期 学期)\\
  卒業研究3(\the\クラス )}
  \end{minipage}
  }

  \XY{\X卒業論文}{\Y卒業論文}{\fontHuge \CoverKintou{6zw}{卒業論文}}

  \XY{\X題目}{\Y題目}{
    \setbox\titlebox=\vbox{
      \begin{minipage}{140mm}\begin{centering}
      \fontHuge\bf \the\卒研題目 \par
      \end{centering}\end{minipage}}
    \vspace*{-\ht\titlebox}
    \box\titlebox%\nointerlineskip
  }
%
  \XY{\X氏名}{\Y氏名}{
  \begin{minipage}{115mm}
  \fontLARGE \begin{list}{}{
  \leftmargin=6zw \labelwidth=6zw \labelsep=5mm \itemsep=-1mm \parsep=4pt}
  \item[\CoverKintou{5zw}{氏名}:]     \the\氏名
  \item[\CoverKintou{5zw}{回生}:]     \the\回生
  \item[\CoverKintou{5zw}{学生証番号}:] \the\学籍番号
  \item[\CoverKintou{5zw}{指導教員}:] \the\指導教員
  \item[\CoverKintou{5zw}{提出日}:]   \the\提出日
  \end{list}
  \end{minipage}}

  \if共同研究者あり
  \XY{\X共同研究者}{\Y共同研究者}{
  \begin{minipage}{115mm}
  \fontLARGE \begin{list}{}{
  \leftmargin=6zw \labelwidth=6zw \labelsep=5mm}
  \item[\CoverKintou{5zw}{共同研究者}:]  \the\共同研究者 \ 
  \end{list}
  \end{minipage}}
  \fi
  \XY{\X情シス}{\Y情シス}{\fontLARGE 立命館大学 情報理工学部 情報システム学科}
%
  \vspace{117.8mm} \hspace{-25mm} \rule{10mm}{0.1mm}
  \newpage
}

%magic way of pointing a place defined by X-Y coordinate
\newdimen\htbackskip
\newdimen\dpbackskip
\long\def\XY#1#2#3{
  \setbox\dummybox=\vbox{
    \vskip#2
    \hbox to \hsize{%
      \hskip#1
      \vbox{\advance\hsize by -#1#3}\hfil
    }
  }
  \htbackskip=-\ht\dummybox
  \dpbackskip=-\dp\dummybox
%
  \box\dummybox\nointerlineskip
  \vskip\htbackskip
  \vskip\dpbackskip
}

\newcommand{\CoverKintou}[2]{\hbox to#1{%
  \kanjiskip=0pt plus 1fill minus 1fill
  \xkanjiskip=\kanjiskip #2}
}
             % 表紙 (年度,タイトル,氏名,学籍番号を記述)
% 内容梗概

\begin{center}
  \large\bf 論文タイトル
\end{center}
\begin{flushright}
名前 書く
\end{flushright}

\vspace*{-12mm}
\section*{\normalsize 内容梗概}
本論文をまとめましょう.

          % 内容梗概(タイトル,氏名を記述)

% 目次(行間を修正して1ページに納めたいなどはこちら)
% {\setlength{\baselineskip}{17.2pt} \tableofcontents}
\tableofcontents            % 目次
\clearpage
\listoffigures              % 図目次
\listoftables               % 表目次
\clearpage

\setcounter{page}{1}
\pagestyle{myheadings}

\section{はじめに}\label{sec:intro}
はじめには,2ページを目安に書きましょう.
卒論は量が多いため,チェックする人は変更点を見つけるのが大変です.
指摘して頂いた点は\verb|\Ca{ }|\Ca{で囲むことで文字が赤色になります}.
提出時(黒に戻すとき)は,main.texの\verb|\setcounter{ChangedColor}{0}|を0から1にしてください.

参考文献はbibtexを使いましょう.普段からゼミで使用している人は,referencesファイルを自分のものに
置き換えてください.
bibtexの使い方は,references.bibを作り,\verb|\cite{jmoni}|の様に本文で参照\cite{jmoni}し,
jbibtexコマンドでさくっとできます.
論文データベースには,必ずbibtex形式というのが用意されているはず.
その内容をコピーすれば基本は大丈夫.
参考文献のスタイルは,情報処理学会の出現順のものを使用しています.
      % はじめに

\section*{謝辞}
\AddTableOfContents{謝辞}

最後に,日頃から励まし,応援して頂いた家族に心より感謝申し上げます.
   % 謝辞

\addcontentsline{toc}{section}{参考文献} %参考文献を目次に入れるやつ

\raggedright
{
	\bibliography{references}
	\bibliographystyle{ipsjunsrt}
}

\end{document}
